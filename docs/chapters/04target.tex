\chapter*{Target Audience}

This manual is designed for anyone interested in exploring or engaging with the exciting field of open-source Application-Specific Integrated Circuit (ASIC) design. While our aims to simplify the complex process of chip layout, some foundational understanding of digital logic and hardware description languages (Verilog), and Python 3 programming language will be beneficial.

Specifically, this manual caters to:

\begin{itemize}
    \item S\textbf{tudents and Educators:} Those learning digital design, VLSI concepts, or exploring the practical aspects of chip fabrication. The tool provides a hands-on platform to apply theoretical knowledge.
    \item \textbf{Hobbyists and Makers:} Individuals eager to design their custom silicon, prototype unique hardware ideas, or simply delve deeper into how chips are made without the prohibitive costs of commercial EDA tools.
    \item \textbf{Researchers:} Academics and industry researchers looking for an open, flexible, and reproducible flow for experimental chip designs, custom hardware acceleration, or exploring novel architectures.
    \item \textbf{Open-Source Hardware Developers:} Contributors to the broader open-source hardware ecosystem who wish to design, verify, and potentially fabricate their own integrated circuits using a transparent and accessible toolchain.
    \item \textbf{Digital Design Engineers:} Professionals with experience in FPGA or traditional ASIC design who are curious about open-source methodologies, seeking to port existing designs, or looking for an alternative flow for specific projects.
\end{itemize}

While we strive to make the process as intuitive as possible, this manual assumes a basic familiarity with command-line interfaces and fundamental digital logic concepts. No prior experience with GDS-II or specific EDA tools is required, as the manual will guide you through the entire process.


\vfill

Happy Designing!

Team Ostival

hello@ostival.org