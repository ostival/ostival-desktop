\chapter{Command Line Interface}\label{ch:cli}
Ostival comes with an integrated Command-Line Interface (CLI), a text-based user interface that enables interaction with the software. The CLI operates through a terminal or console window, where users can type commands to execute specific tasks.

\section{Command Usage}
\begin{verbatim}
Ostival [option] <commands>
\end{verbatim}

\section{General}
The Ostival general command is used to check the status of the software.
\subsection{help}
This command lists the command and provides a brief description of its usage.
\begin{verbatim}
Ostival -h or Ostival -? or Ostival --help
\end{verbatim}

\subsection{version}
This command displays the current version of the Ostival software.
\begin{verbatim}
Ostival -v or Ostival --version
\end{verbatim}

\subsection{proj}
This command displays the description of the current project created in the Ostival software.
\begin{verbatim}
Ostival -p or Ostival --proj
\end{verbatim}

\subsection{tool}
This command lists the tools required by Ostival and indicates which tools are installed in the system.
\begin{verbatim}
Ostival -t or Ostival --tool
\end{verbatim}

